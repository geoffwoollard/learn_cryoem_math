\documentclass[11pt, oneside]{article}   	% use "amsart" instead of "article" for AMSLaTeX format
\usepackage{geometry}                		% See geometry.pdf to learn the layout options. There are lots.
\geometry{letterpaper}                   		% ... or a4paper or a5paper or ... 
%\geometry{landscape}                		% Activate for rotated page geometry
%\usepackage[parfill]{parskip}    		% Activate to begin paragraphs with an empty line rather than an indent
\usepackage{graphicx}				% Use pdf, png, jpg, or eps§ with pdflatex; use eps in DVI mode
								% TeX will automatically convert eps --> pdf in pdflatex		
\usepackage{amssymb}
\usepackage{hyperref}
\usepackage{multirow}
\usepackage{ulem}

%SetFonts

%SetFonts


\title{Online CryoEM Study Group}
\author{Geoffrey Woollard, Toronto, Canada}
%\date{}							% Activate to display a given date or no date

\begin{document}
\maketitle

\section{General Information}

In just a few days since my message on 3DEM there were over 100 responses to the survey. Below is the plan, which will kick off this upcoming week on {\bf Thursday 19 Nov at 9 AM PST. Dr. Frederick Sigworth will be our special guest for this inaugural meeting.} Now is the time to extend invitations far and wide. Feel free to share this document and direct people to sign up at \url{https://forms.gle/BUeUW14vV4pyQbDDA} so I have the emails in one place. Online meeting link to follow (Zoom). Forgive typographical errors for this document, which I wanted to get out as soon as possible. {\bf Please join the Slack group and ask questions there, rather than emailing me, as there are too many of us and I will be overwhelmed!}

\subsection{Meeting Frequency and Length}
The most popular meeting frequency was every 1-2 weeks (59) followed by once a month (32). We'll hit the ground running and start off every 1-2 weeks. Depending on how things go we way move to once a month. For length the popular times were 60-90 min (47), followed closely by 30-45 min (41). Thus the initial format will be:
\begin{itemize}
	\item Every 1-2 weeks
	\item 60-90 min
\end{itemize}

\subsection{Audience and Streams}
Over 2/3 of the audience are grad students. There rest of the audience is a mix of long term staff (13, facility manager, seniour scientist, research associate, unspecified industry position, etc.), principal investigators (4, industry or academic), postdocs (9), and undergraduate (1). After carefully reading the responses, I think it would be best to have two streams each meeting. I have code named them the {\bf blur} and {\bf sharpen} stream, referring to map sharpening. I am counting on your sense of humour and humility for this playful naming schema! Perhaps its a bit more gentle than low and high resolution? Remember that both blurred and sharpened maps are important for being able to see what is going on and build a model in to the map. The goal of this study group is to have both intuition and be able to connect that back to the math, never missing the forest for the trees.

{\bf  Blur stream}: Beginners and intermediates with may have years of expertise doing sample prep, collecting data on the scope, making maps, and building models. However, many people with such expertise expressed a desire to go deeper into the fundamentals, and understand things better under the hood, and requested help to become more confident in the math. One person put it bluntly, "I am interested in being more than a button pusher."

{\bf Sharpen stream}: This stream can be for people with a high degree of expertise that are looking to sharpen their skills, and interact with other experts. For example: methods developers with perhaps not that much experience processing datasets from start to finish, university instructors, facility managers who have a strong working knowledge but want to go deeper into the foundations and refresh, people with strong math and physics backgrounds that are fairly new but will catch on quickly, experts in crystallography that are switching over to cryoEM but that have a strong theoretical basis in the underlying math already.

After reading through the responses a few times, I would estimate the ratio of the streams to be about 3/4 blur to 1/4 sharpen. In the first meeting you can choose your own stream, and go to breakout rooms with this in mind. I will make multiple breakout rooms so that there can be small group discussion of 4-8 people. If you are an expert feel free to go to the blur stream and help out. If you judge yourself an intermediate-beginner, you can test the waters in the sharpen stream. The two stream format may evolve over time (e.g. three streams, a lecture with everyone together), but let's try out two streams and see how it goes.

\subsection{Meeting Format}
We will start with the "flipped classroom" and "breakout room discussions" to get things off the ground with minimal overhead. After some initial meetings we will try out a mix of the following formats listed below, with their surveyed popularity  listed next to them. {\bf If you would like to volunteer to give a lecture or be available for office hours then please send me an email to volunteer yourself.}
\begin{itemize}
	\item Lecture with Q\&A (72)
	\item Flipped classroom: i.e. we have a syllabus with pre-reading (textbook chapter, review paper) that we go through before and then discuss online (57)
	\item Breakout room discussion (small groups organized thematically) (35)
	\item Office hours with an expert available to answer questions (24)	
\end{itemize}

\section{Dates and Topics}

\begin{center}
\small
 \begin{tabular}{|| c c c c||} 
 \hline
 Date & Time & Stream & Topic \\ [0.5ex] 
 \hline\hline
Th 19 Nov 2020 & \tiny{9 AM PST / 12 PM EST} &  blur/sharpen & defocus phase contrast  \\ 
 \hline
\sout{Th 26 Nov 2020}  &   & & (cancelled for US Thanksgiving )    \\ 
 \hline
Th 3 Dec 2020 & \tiny{9 AM PST / 12 PM EST} & blur & Fourier transform   \\ 
 \hline
Th 10 Dec 2020 & \tiny{9 AM PST / 12 PM EST} & sharpen & Fourier transform   \\ 
 \hline
Th 17 Dec 2020 & \tiny{9 AM PST / 12 PM EST} & blur & convolution, sampling, Nyquist   \\ 
 \hline
Th 7 Jan 2021 & \tiny{9 AM PST / 12 PM EST} & sharpen & convolution, sampling, Nyquist   \\ 
 \hline
Th 17 Dec 2020 & \tiny{9 AM PST / 12 PM EST} & blur & phase-contrast in the EM   \\ 
 \hline
Th 7 Jan 2021 & \tiny{9 AM PST / 12 PM EST} & sharpen & phase-contrast in the EM   \\ 
 \hline
Th 21 Jan 2021 & \tiny{9 AM PST / 12 PM EST} & blur & 2D Expectation-maximization   \\ 
 \hline
Th 28 Jan 2021 & \tiny{9 AM PST / 12 PM EST} & sharpen & 2D Expectation-maximization   \\ 
 \hline
Th 4 Feb 2021 & \tiny{9 AM PST / 12 PM EST} & blur & 2D Expectation-maximization   \\ 
 \hline
Th 11 Jan 2021 & \tiny{9 AM PST / 12 PM EST} & sharpen & 2D Expectation-maximization   \\ 
 \hline
Th 18 Feb 2021 & \tiny{9 AM PST / 12 PM EST} & blur & image formation (forward model)   \\ 
 \hline
Th 25 Jan 2021 & \tiny{9 AM PST / 12 PM EST} & sharpen & omage formation (forward model), multislice  \\  
 \hline
Th 21 Jan 2021 & \tiny{9 AM PST / 12 PM EST} & ... & Wah Chiu et al, lectures, ...  \\ [1ex] 
 \hline
 ... 2021 ... & ... &  ... & ...  \\ 
 \hline
\end{tabular}
\end{center}

\section{Meeting Time for a Global Audience}
About 94\% of those interested in North America and Europe (mainly UK and Germany) (85), with the rest in Israel, China, India, Bangladesh, Australia (5). For now we will have the time at 9 AM PST, thus 7 PM in Israel. I am open to proposals to occasionally have a different time to accommodate a global audience - especially if the presenter is from a less represented time zone. I tried this \href{https://www.timeanddate.com/worldclock/meeting.html}{online time zone tool} and it didn't suggest anything that worked. Suggestions are welcome for this aspect.

For the first meeting we will meet at 9 AM PST, which is 12 PM EST, 5 PM GMT, 7 PM GMT+2, 1 AM next day GMT+8, 3 AM next day GMT+10.

\section{Organizational Team}
To the question 'Would you like to be part of the organizational team?' 39 people answered maybe and 4 people answered yes.

I would appreciate help 
\begin{itemize}
	\item {\bf Giving lectures. You can send a proposal to me including topic, format, time, date.}
	\item Developing the syllabus, and suggesting (or making) learning content.
	\item Formulating challenge problems to test comprehension, build intuition, and conceptual understanding.
	\item {\bf Being the leader of a breakout room to answer questions. You could list your areas of expertise, so people would know to come to you for those things. A sort of 'office hours'.}
	\item Zoom co-host.
	\item Suggesting improvements to me. You could ask around and see how people are finding things.
\end{itemize}

\section{Slack}
We will use the Slack channel 'cryoem\_study\_group' for asynchronous chat. The link to join is 
{\tiny \url{https://join.slack.com/t/cryoemstudygroup/shared\_invite/zt-j66wuws3-~UcfsdmtQow~7qYC~iJu\_g}}. Please join the Slack group and ask questions there, rather than emailing me, as there are too many of us and I will be overwhelmed!

\section{Syllabus}
We will start off drawing heavily from the content developed by Dr Frederick Sigworth, \url{https://cryoemprinciples.yale.edu/video-lectures}. 

If you haven't already gone through Grant Jensen's \href{https://www.caltech.edu/about/news/grant-jensen-cryo-em}{popular} online course \href{https://jensenlab.caltech.edu/courses/}{'Getting Started in Cryo-EM' }, now would be a good time to do so. I think enough people have gone through this on their own that we will mainly draw from other material. 

\pagebreak
\subsection{19 Nov 2020}
\subsubsection{Blur}
\textendash {Pre-reading}
\begin{itemize}
\item \href{https://yale.app.box.com/s/frs9qtm28lsb1cts9bvxsh9nn6m6zbzv}{Complex numbers and the complex exponential} (10 min)
\item \href{https://cryoemprinciples.yale.edu/sites/default/files/files/1%20Review%20of%20Complex%20Numbers.pdf}{Review	of complex	numbers} (3 pages)
\item \href{https://yale.app.box.com/s/2nciyzqd576e61kpopi7zdd08gsl35zz}{Defocus phase contrast} (35 min)

\end{itemize}
\textendash {Questions}
\begin{itemize}
\item What is physically happening to the sample when the electron is detected at small diffraction angles vs large diffraction angles? What else can happen to the electron?
%Watch video 2.1 (10 min). Towards the end (7:40 mark) Dr. Sigworth makes some interesting 2D/3D plots showing the real and imaginary part of exp(i theta). Code up a way to plot this and share it with everyone. For example, you could publish a coding notebook on something like GitHub.
\end{itemize}

\subsubsection{Sharpen}
\textendash{Pre-reading}
\begin{itemize}
	\item \href{https://yale.app.box.com/s/2nciyzqd576e61kpopi7zdd08gsl35zz}{Defocus phase contrast} (35 min)
	\item 6.2 The Wave Equation for Fast Electrons in 'Advanced Computing in Electron Microscopy', Kirkland (2000), pp. 156-159.
\end{itemize}
\textendash {Questions}
\begin{itemize}
\item In Sigworth's derivation of $|\Psi|^2$ in the 'Defocus phase contrast' video, he made various assumptions such as small theta, small epsilon. Under what extreme conditions would they break down. Do these occur in cryoEM? In other experimental regimes besides what is typical in cryoEM?
\item Around the 23-25 min mark of the 'Defocus phase contrast' video the CTF seems to oscillate to zeros. Is this observed in practice? Why or why not?
\item Work through the derivation with pencil and eraser, justifying each step as best you can. come with your questions to the group study.
\item Biological samples are made of atoms that give faint contrast, when compared to samples with higher atomic numbers. Where does the atomic number of the sample come into the equations presented here? 
\end{itemize}

\pagebreak
\subsection{26 Nov 2020}
\subsubsection{Blur}
\textendash {Pre-reading}
\begin{itemize}
	\item \href{https://yale.app.box.com/s/qzgp41qznzw0k6gg5wqsx2uwmpz2n1gx}{The Fourier transform in one dimension} (35 min)
	\item \href{https://cryoemprinciples.yale.edu/sites/default/files/files/3%20Fourier1D.pdf}{The 1D Fourier Transform} (7 pages)
	\item Interactive coding notebook \href{https://gitlab.tudelft.nl/aj-lab/teaching/-/wikis/NB4020}{'Practical 1 - Fundamentals of Image Processing'} in the High-resolution imaging course at TUDelft. Note there are multiple notebooks: Introduction, Fourier Series, Frequency Spectrum, 2-D Fourier Analysis. See the links at the bottom.
\end{itemize}
\textendash {Questions}
\begin{itemize}
	\item Assume you have a 128 \AA box size, with pixel size of 1 \AA. If the spacing between bins in Fourier space is *, and there are 64 bins in the negative direction and 65 bins in the positive direction, what length ranges (in units of \AA) do the first few and last few frequency bins correspond to?  How many frequency bins are between 10 and 5 \AA, versus 5 and 2.5 \AA? \\ Hint: $\langle[..., [0,1/128), [1/128,2/128), ... , [\frac{128/2-2}{128}, \frac{128/2-1}{128}),  [\frac{128/2-1}{128}, \frac{128/2}{128})\rangle$.
\end{itemize}

\subsubsection{Sharpen}
\textendash {Pre-reading}
\begin{itemize}
\item \href{https://yale.app.box.com/s/5kaneezyqqzl08w8gtt62z90apbo8u1p}{The Fourier transform in two and three dimensions} (43 min)
	\item \href{https://cryoemprinciples.yale.edu/sites/default/files/files/4%20Fourier2D-3D.pdf}{2D and 3D Fourier transforms} (9 pages)
	\item Interactive coding notebook \href{https://gitlab.tudelft.nl/aj-lab/teaching/-/wikis/NB4020}{'Practical 1 - Fundamentals of Image Processing'} in the High-resolution imaging course at TUDelft. Note there are multiple notebooks: Frequency Spectrum, 2-D Fourier Analysis. See the links at the bottom.

	
\end{itemize}
\textendash{Questions}
\begin{itemize}
	\item Sample a simple function (e.g. $\exp$, $\sin$, gaussian) in 1D or 2D. Then use some library to compute the FFT. Then compute the DFT in your own implementation. How close is the error? Work our the solution analytically for the continuous case. What should the answer be at some discrete points according to the continuous case, and how is it different from what the FFT gave? What is the typical floating point error? What was the speedup of the FFT?
\end{itemize}

\pagebreak
\subsubsection{3 Dec 2020}
\subsubsection{Blur}
\textendash {Pre-reading}
\begin{itemize}
	\item \href{https://yale.app.box.com/s/00d0q34y1ef8qdfw3peo0r2jvcdf3hb3}{Fourier transform: convolution, sampling and Nyquist} (37 min)
	\item Interactive coding notebook \href{https://gitlab.tudelft.nl/aj-lab/teaching/-/wikis/NB4020}{'Practical 1 - Fundamentals of Image Processing'} in the High-resolution imaging course at TUDelft. Note there are multiple notebooks: 2-D Fourier Analysis, Convolutions. See the links at the bottom.

\end{itemize}
%\textendash {Questions}
%\begin{itemize}
%	\item When does convolution (physically or computationally) happen in cryoEM happen? What is being convolved, what is it's functional form, and how many dimensions are involved?
%\end{itemize}

\subsubsection{Sharpen}
\textendash{Pre-reading}
\begin{itemize}
	\item \href{https://yale.app.box.com/s/00d0q34y1ef8qdfw3peo0r2jvcdf3hb3}{Fourier transform: convolution, sampling and Nyquist} (37 min)
	\item Interactive coding notebook \href{https://gitlab.tudelft.nl/aj-lab/teaching/-/wikis/NB4020}{'Practical 1 - Fundamentals of Image Processing'} in the High-resolution imaging course at TUDelft. Note there are multiple notebooks: 2-D Fourier Analysis, Convolutions. See the links at the bottom.

\end{itemize}
%\textendash {Questions}
%\begin{itemize}
%	\item Code up a simple example illustrating the convolution theorem, where you also actually do the convolution in real space. How close is it to the answer where you did the multiplication in Fourier space? What was the speedup? Now try speeding up the calculation by of the convolution in real space by making the convolution kernel smaller. How good of an approximation is this? Use a 2D projection of a 3D density map and a meaningful kernel (e.g. low pass filter) to build your intuition in a useful way.
%\end{itemize}

\pagebreak
\subsection{10 Dec 2020}
\subsubsection{Blur}
\textendash{Pre-reading}
\begin{itemize}
	\item \href{https://cryoemprinciples.yale.edu/sites/default/files/files/2%20Phase%20contrast.pdf}{Phase-contrast imaging in the EM'} (10 pages)
	\item Interactive coding notebook \href{https://gitlab.tudelft.nl/aj-lab/teaching/-/wikis/NB4020}{'Practical 1 - Fundamentals of Image Processing'} in the High-resolution imaging course at TUDelft. Note there are multiple notebooks: CTF. See the links at the bottom.

\end{itemize}
%\textendash {Questions}
%\begin{itemize}
%	\item See the  \href{https://static5.olympus-lifescience.com/data/olympusmicro/primer/images/mtf/modulationfigure6.jpg?rev=480E}{'signal star'} image on panel (b).
%What would this image look like if the CTF was applied (i.e. through convolution)?
%	\item As defined in equation 11, what happens when the B factor is negative versus positive?
%\end{itemize}

\subsubsection{Sharpen}
\textendash{Pre-reading}
\begin{itemize}
	\item  \href{https://cryoemprinciples.yale.edu/sites/default/files/files/2%20Phase%20contrast.pdf}{Phase-contrast imaging in the EM'} (10 pages)
	\item Interactive coding notebook \href{https://gitlab.tudelft.nl/aj-lab/teaching/-/wikis/NB4020}{'Practical 1 - Fundamentals of Image Processing'} in the High-resolution imaging course at TUDelft. Note there are multiple notebooks: CTF. See the links at the bottom.
\end{itemize}
%\textendash {Questions}
%\begin{itemize}
%	\item In practice, on the scope, the CTF is easier to see in an FFT if you are at high mag. Why? Can you relate this back to the math? How is magnification playing into the equation?
%	\item As defined in equation 11 of the 'Phase-contrast imaging in the EM' pdf, what physical behaviour is involved in this B factor? What other factors of B are there in cryoEM and what physical behaviour is involved in those B factors? Why are they all called B-factors?
%\end{itemize}

\pagebreak
\subsection{17 Dec 2020}
\subsubsection{Blur}
\textendash{Pre-reading}
\begin{itemize}
	\item \href{https://repository.upenn.edu/cgi/viewcontent.cgi?article=1665&context=physics_papers}{Nelson, P. C. (2019). Chapter 12 : Single Particle Reconstruction in Cryo-electron Microscopy. In Physical Models of Living Systems (pp. 305?325).}
	\item Interactive coding notebook \href{https://gitlab.tudelft.nl/aj-lab/teaching/-/wikis/NB4020}{'Practical 2 - 2D/3D reconstruction'} in High-resolution imaging course at UTDelft.
\end{itemize}
%\textendash {Questions}
%\begin{itemize}
%	\item Some popular cryoEM software does 2D classification in Fourier space. Do these equations work in Fourier space? What would have to be changed?
%	\item In Nelson's treatment, there was only one class. To do multiple classes, what would have to be changed? What is happening during a typical 2D classification 'round'? What is constant during a round and what is updated between rounds?
%\end{itemize}

\subsubsection{Sharpen}
\textendash{Pre-reading}
\begin{itemize}
	\item \href{https://repository.upenn.edu/cgi/viewcontent.cgi?article=1665&context=physics_papers}{Nelson, P. C. (2019). Chapter 12 : Single Particle Reconstruction in Cryo-electron Microscopy. In Physical Models of Living Systems (pp. 305?325).}
	\item Interactive coding notebook \href{https://gitlab.tudelft.nl/aj-lab/teaching/-/wikis/NB4020}{'Practical 2 - 2D/3D reconstruction'} in the High-resolution imaging course at TUDelft.
\end{itemize}
%\textendash {Questions}
%\begin{itemize}
%	\item How does a prior affect the Log loss function? What functional forms of the priors are convenient for optimization, and what is their interpretation in terms of probability?
%	\item It can be shown that iid Gaussian additive noise in real space is also Gaussian additive noise in Fourier space. See \href{https://dsp.stackexchange.com/questions/24170/what-are-the-statistics-of-the-discrete-fourier-transform-of-white-gaussian-nois}{this dsp.stackexchange.com post} for example. Try to generalize this somehow. What happens for iid Gaussian additive noise that is radially dependent in Fourier space, or pixel dependent in real space?
%\end{itemize}

\pagebreak
\subsection{7 Jan 2020 2020}
\subsubsection{Blur}
\textendash{Pre-reading}
\begin{itemize}
	\item \href{https://youtu.be/tzv5c5K7MEk?t=4690}{NCCAT SPA short course 2020, Lecture 4: Algorithms and foundational math Part I \& 2, Fred Sigworth} (1:18:10 - 1:28:46, ~10 min)
	\end{itemize}
%\textendash {Questions}
%\begin{itemize}
%	\item How is real noise different from Gaussian additive noise? What makes it different? If particles come from different types of grids (thin continuous carbon, single layer graphene) can they be combined into one large dataset to solve a structure? Why don't researchers pool different datasets of the same particle (e.g. from different EMPIAR entries) to solve a high resolution structure with tens of millions of particles?
%\end{itemize}

\subsubsection{Sharpen}
\textendash{Pre-reading}
\begin{itemize}
	\item Section '2. Theory' in Vulovi\'{c}, M., Ravelli, R. B. G., van Vliet, L. J., Koster, A. J., Lazi\'{c}, I., L\"ocken, U., ? Rieger, B. (2013). Image formation modeling in cryo-electron microscopy. Journal of Structural Biology, 183(1), 19?32. \url{http://doi.org/10.1016/j.jsb.2013.05.008}
	\item Supplementary material associated with the article 'Image Formation Modeling in Cryo-Electron Microscopy' (18 pages)
	\item 6.4 The Multislice Solution in 'Advanced Computing in Electron Microscopy', Kirkland (2000), pp. 162-165
\end{itemize}
%\textendash {Questions}
%\begin{itemize}
%	\item Equation 7 in Vulovi\'{c} et al. shows the DQE/NTF being applied. In practice, the DQE/MTF/NTF is often plotted in 1D in papers. What would be the difference between doing the convolution of the DQE/NTF in 2D versus 1D?
%	\item Eqation 5 in Vulovi\'{c} et al. shows the 'complex CTF'. What is the difference between this and the $\sin$ CTF that is often used? When is it appropriate to use one versus the other treatment?
%\end{itemize}

\pagebreak
\subsection{21 Jan 2020}
By this time the anticipated textbook by Wah Chiu, Robert Glaeser, and Eva Nogales will hopefully be out, and we can see how it looks. We could also plan some lectures.

\end{document}  